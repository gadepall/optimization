\documentclass[journal,10pt,twocolumn]{article}
\usepackage{graphicx, float}
\usepackage[margin=0.5in]{geometry}
\usepackage{amsmath, bm}
\usepackage{array}
\usepackage{booktabs}
\usepackage{mathtools}

\providecommand{\norm}[1]{\left\lVert#1\right\rVert}
\let\vec\mathbf
\newcommand{\myvec}[1]{\ensuremath{\begin{pmatrix}#1\end{pmatrix}}}
\newcommand{\mydet}[1]{\ensuremath{\begin{vmatrix}#1\end{vmatrix}}}

\title{\textbf{Optimization Assignment}}
\author{Maddu Dinesh}
\date{September 2022}

\begin{document}

\maketitle
\paragraph{\textit{Problem Statement} -At what points in the interval (0,2$\pi$) does the function sin2x attain its maximum value .}

\section*{\large Figure}

\begin{figure}[H]
\centering
\includegraphics[width=1\columnwidth]{a.png}
\caption{Graph of f(x)}
\label{fig:triangle}
\end{figure}
\section*{\large Solution}

	
    \subsection*{\normalsize Gradient descent}
    
    
    \begin{align}
	\label{eq:vol_varx}
	f(x) = sin2x\\
    f'(x) = 2cos2x
	\end{align}

we have to attain the maximum value of sin2x in the interval [0,2$\pi$]. This can be seen in Figure f(x).Using gradient ascent method we can find its maxima in the interval [0,2$\pi$]
\begin{equation}
        x_{n+1} = x_n + \alpha \nabla f(x_n) \\
\end{equation}
\vspace{1mm}
\begin{equation}
\implies x_{n+1}=x_n+\alpha(2cos2x)
\end{equation}

Taking $x_0=0.5,\alpha=0.001$ and precision = 0.00000001, values obtained using python are:
    

    \begin{align}
        \boxed{\text{Maxima} = 1.0000}\\
        \boxed{\text{Maxima Point} = 0.7854}
    \end{align}
   
    

    





 






\end{document}
