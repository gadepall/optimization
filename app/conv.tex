%\begin{enumerate}[label=\thechapter.\arabic*,ref=\thechapter.\theenumi]
%\numberwithin{equation}{enumi}
%\numberwithin{figure}{enumi}
%\numberwithin{table}{enumi}
\item
	\label{prop:app/convconvconvex_def}
A single variable function $f$ is said to be convex if
\begin{align}
	\label{eq:app/convconvconvex_def}
	f\sbrak{\mu x_1 + \brak{1-\mu}x_2} \leq \mu f\brak{x_1} + \brak{1-\mu}f\brak{x_2},
\end{align}
for $\quad 0 < \mu < 1$ and $x_1, x_2 \in \mathbb{R}$.

For a generic quadratic function $ax^2+bc+c$, let us determine the sufficient condition for it to be convex. Let 
\begin{align}
	\label{eq:app/convconvEq1}
	f\brak{x} &= ax^2+bx+c 
\end{align}
Substituting LHS of inequality from \eqref{eq:app/convconvconvex_def} in \eqref{eq:app/convconvEq1}
\begin{multline}
   \label{eq:app/convconvEq2}
	f\sbrak{\mu x_1 + \brak{1-\mu}x_2}  = f\sbrak{x_2 + \mu \brak{x_1-x_2}}\\ 
	   \implies  a\sbrak{x_2+\mu\brak{x_1-x_2}}^2 b\sbrak{x_2+\mu\brak{x_1-x_2}} + c \\ 
	   \implies ax_2^2 + a\mu^2 x_1^2+a\mu^2 x_2^2 - 2a\mu^2 x_1x_2 \\ 
	   +2a\mu x_1x_2 - 2a\mu x_2^2+bx_2+b\mu x_1-b\mu x_2+c 
\end{multline} 
Substituting RHS of inequality from \eqref{eq:app/convconvconvex_def} in \eqref{eq:app/convconvEq1}
\begin{multline}
	\mu f\brak{x_1} + \brak{1-\mu}f\brak{x_2}  = a \mu x_1^2 + b\mu x_1 + \mu c \\ 
		 + \brak{1-\mu}\brak{ax_2^2+bx_2+c} \\
	\label{eq:app/convconvEq3}
		\implies  a \mu x_1^2 + b\mu x_1 + ax_2^2 + bx_2 + c -a\mu x_2^2 -b\mu x_2
\end{multline} 
Combining \eqref{eq:app/convconvEq2} and \eqref{eq:app/convconvEq3} with inequality and simplifying
\begin{multline}
a\mu^2 x_1^2 + a\mu^2 x_2^2 - 2a\mu^2 x_1x_2 +2a\mu x_1x_2 - 2a\mu x_2^2 \leq \\
	a\mu x_1^2  -a\mu x_2^2  
\end{multline} 
\begin{multline}
	\label{eq:app/convconvEq4}
	\implies a\mu^2 x_1^2 + a\mu^2 x_2^2 - 2a\mu^2 x_1x_2 +2a\mu x_1x_2 - a\mu x_2^2 - a\mu x_1^2 \leq 0 \\ 
	\implies   x_1^2\brak{a\mu^2-a\mu} + x_2^2\brak{a\mu^2 - a\mu} - 2x_1x_2\brak{a\mu^2-a\mu} \leq 0 \\ 
	 \implies  \brak{a \mu^2-a\mu}\brak{x_1-x_2}^2 \leq 0\\
	 \implies  a\mu\brak{1-\mu}\brak{x_1-x_2}^2 \geq 0
\end{multline}
For the inequality in \eqref{eq:app/convconvEq4} to be true,
\begin{align}
	a \geq 0 \because \mu, 1-\mu \geq 0, \brak{x_1-x_2}^2 \geq 0
\end{align}
However, $a \neq 0$, since it is a quadratic function. Hence $a > 0$, for $f\brak{x}$ to be convex.
\item The quadratic form
        \begin{align}
            q\brak{\vec{x}} \triangleq \vec{x}^\top\vec{Ax} + \vec{b}^\top\vec{x} + c
            \label{eq:quad-x}
        \end{align}
        is convex iff $\vec{A}$ is positive semi-definite.
	\\
	\solution
	Consider two points $\vec{x_1}$ and $\vec{x_2}$, and a real constant
        $0 \le \mu \le 1$. Then,
        \begin{align}
            &\mu f\brak{\vec{x_1}} + \brak{1-\mu}f\brak{\vec{x_2}} - f\brak{\mu\vec{x_1}+\brak{1-\mu}\vec{x_2}} \nonumber \\
            &= \brak{\mu-\mu^2}\vec{x_1}^\top\vec{Ax_1} + \brak{1-\mu-\brak{1-\mu}^2}\vec{x_2}^\top\vec{Ax_2} \nonumber \\
            &- 2\mu\brak{1-\mu}\vec{x_1}^\top\vec{Ax_2} \\
            &= \mu\brak{1-\mu}\brak{\vec{x_1}^\top\vec{Ax_1}-2\vec{x_1}^\top\vec{Ax_2}+\vec{x_2}^\top\vec{Ax_2}} \\
            &= \mu\brak{1-\mu}\brak{\vec{x_1}-\vec{x_2}}^\top\vec{A}\brak{\vec{x_1}-\vec{x_2}}
            \label{eq:psd-iff}
        \end{align}
        Since $\vec{x_1}$ and $\vec{x_2}$ are arbitrary, it follows from 
        \eqref{eq:psd-iff} that
        \begin{align}
            \mu f\brak{\vec{x_1}} + \brak{1-\mu}f\brak{\vec{x_2}} \ge f\brak{\mu\vec{x_1}+\brak{1-\mu}\vec{x_2}}
        \end{align}
        iff $\vec{A}$ is positive semi-definite, as required.

\item     Let
	\begin{align}
        \vec{M} &\triangleq \myvec{\vec{m_1} & \vec{m_2}} \label{eq:app-M-def} \\
        \bm{\mu} &\triangleq \myvec{\mu_1\\-\mu_2} \label{eq:app-mu-def} \\
    \end{align}
	The function 
	\begin{align}
        f\brak{\bm{\mu}} \triangleq 
                               \norm{\vec{M}\bm{\mu}-\vec{x}}^2 
        \label{eq:app-f-def}
    \end{align}
    is convex.
    \\
    \solution
        \eqref{eq:app-f-def} can be expressed as
\begin{align}
	\brak{\vec{M}\bm{\mu}-\vec{x}}^\top\brak{\vec{M}\bm{\mu}-\vec{x}}
\end{align}
Consider $\bm{\mu}_1$ and 
    $\bm{\mu}_2$ and let $0 \le \mu \le 1$. Then,
    \begin{align}
        f\brak{\mu\bm{\mu}_1+\brak{1-\mu}\bm{\mu}_2} 
        &= \norm{\vec{M}\brak{\mu\bm{\mu}_1+\brak{1-\mu}\bm{\mu}_2}-\vec{x}} \\
        &= \norm{\mu\brak{\vec{M}\bm{\mu}_1-\vec{x}}+\brak{1-\mu}\brak{\vec{M}\bm{\mu}_2-\vec{x}}} \\
	    &\le \mu\norm{\vec{M\bm{\mu}_1}-\vec{x}} + \brak{1-\mu}\norm{\vec{M\bm{\mu}_2}-\vec{x}}
        \label{eq:app-convex-ineq}
    \end{align}
    Where \eqref{eq:app-convex-ineq} follows from the triangle inequality.
    \item Show that the  quadratic programming problem
	    \label{app:quad-nonconv}
 \begin{align}
        \min_{\vec{x}} g\brak{\vec{x}} &= \norm{\vec{x}-\vec{P}}^2 \\
        \textrm{s.t. } h\brak{\vec{x}} &= \vec{x}^\top\vec{Vx} + 2\vec{u}^\top\vec{x} + f = 0, \quad \vec{V}\succeq \vec{0} \label{eq:12/6/5/27/conv/conv/nonconv-constr/app}
    \end{align}
    is nonconvex.
    \\
    \solution  Let $\vec{x_1}$ and $\vec{x_2}$ satisfy \eqref{eq:12/6/5/27/conv/conv/nonconv-constr/app}. Then,
    \begin{align}
       h\brak{\vec{x}_1}= \vec{x_1}^\top\vec{Vx_1} + 2\vec{u}^\top\vec{x_1} + f &= 0 \label{eq:12/6/5/27/conv/conv/x1-parab} \\
 h\brak{\vec{x}_2} =       \vec{x_2}^\top\vec{Vx_2} + 2\vec{u}^\top\vec{x_2} + f &= 0 \label{eq:12/6/5/27/conv/conv/x2-parab}
    \end{align}
    Then, for any $0 \le \mu \le 1$,  from
            \eqref{eq:psd-iff},
 \begin{align}
            \mu h\brak{\vec{x_1}} + \brak{1-\mu}h\brak{\vec{x_2}} =  h \brak{\mu\vec{x_1}+\brak{1-\mu}\vec{x_2}} 
            + \mu\brak{1-\mu}\brak{\vec{x_1}-\vec{x_2}}^\top\vec{A}\brak{\vec{x_1}-\vec{x_2}}
            \label{eq:psd-iff-quad}
        \end{align}
From \eqref{eq:12/6/5/27/conv/conv/x1-parab}  and \eqref{eq:12/6/5/27/conv/conv/x2-parab},
    \begin{align}
	    \mu h\brak{\vec{x_1}} + \brak{1-\mu}h\brak{\vec{x_2}} = 0
\label{eq:12/6/5/27/conv/conv/nonconv-constr/app/hx}
    \end{align}
    which, upon substituting in 
            \eqref{eq:psd-iff-quad}
	    yields
    \begin{align}
 h \brak{\mu\vec{x_1}+\brak{1-\mu}\vec{x_2}} 
            + \mu\brak{1-\mu}\brak{\vec{x_1}-\vec{x_2}}^\top\vec{A}\brak{\vec{x_1}-\vec{x_2}} = 0
	    \\
	    \implies  h \brak{\mu\vec{x_1}+\brak{1-\mu}\vec{x_2}} \le 0
    \end{align}
    since 
\begin{align}
             \mu\brak{1-\mu}\brak{\vec{x_1}-\vec{x_2}}^\top\vec{A}\brak{\vec{x_1}-\vec{x_2}} > 0
\end{align}
     Hence, the optimization 
    problem is nonconvex as the set of points on the curve do not form a 
    convex set.
    \item If $\vec{P}$ lies \textit{outside} the given curve, show that the following
    relaxation makes the above problem convex.
	    \label{app:quad-conv}
    \begin{align}
        \min_{\vec{x}} g\brak{\vec{x}} &= \norm{\vec{x}-\vec{P}}^2 \label{eq:12/6/5/27/conv/conv/cost} \\
        \textrm{s.t. } h\brak{\vec{x}} &= \vec{x}^\top\vec{Vx} + 2\vec{u}^\top\vec{x} + f \le 0, \quad \vec{V}\succeq \vec{0} \label{eq:12/6/5/27/conv/conv/constr}
    \end{align}
    \solution 
    In this case, 
    \begin{align}
	    \mu h\brak{\vec{x_1}} + \brak{1-\mu}h\brak{\vec{x_2}} \le 0
\label{eq:12/6/5/27/conv/conv/conv-constr/app/hx}
\end{align}
and from 
            \eqref{eq:psd-iff-quad},
 \begin{align}
 h \brak{\mu\vec{x_1}+\brak{1-\mu}\vec{x_2}} 
            + \mu\brak{1-\mu}\brak{\vec{x_1}-\vec{x_2}}^\top\vec{A}\brak{\vec{x_1}-\vec{x_2}} \le 0
	    \\
	    \implies  h \brak{\mu\vec{x_1}+\brak{1-\mu}\vec{x_2}} \le 0
    \end{align}
    Hence, the optimization problem is convex. 

\end{enumerate}

