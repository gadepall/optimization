\iffalse
\documentclass[12pt]{article}
\usepackage{graphicx}
\usepackage{amsmath}
\usepackage{mathtools}
\usepackage{gensymb}
\usepackage{tabularx}
\usepackage{array}
\usepackage[latin1]{inputenc}
\usepackage{fullpage}
\usepackage{color}
\usepackage{array}
\usepackage{longtable}
\usepackage{calc}
\usepackage{multirow}
\usepackage{hhline}
\usepackage{ifthen}
\usepackage{lscape}
\usepackage{float}
\usepackage{amssymb}

\newcommand{\mydet}[1]{\ensuremath{\begin{vmatrix}#1\end{vmatrix}}}
\providecommand{\brak}[1]{\ensuremath{\left(#1\right)}}
\providecommand{\norm}[1]{\left\lVert#1\right\rVert}
\providecommand{\abs}[1]{\left\vert#1\right\vert}
\newcommand{\solution}{\noindent \textbf{Solution: }}
\newcommand{\myvec}[1]{\ensuremath{\begin{pmatrix}#1\end{pmatrix}}}
\let\vec\mathbf

\def\inputGnumericTable{}

\begin{document}
\begin{center}
\textbf\large{OPTIMIZATION}

\end{center}
\section*{Excercise 6.6}

\solution
\fi
The given equation of the curve can be written as  
\begin{align}
	\label{eq:12/6/6/4/sdp/parabolaEq2}
	g\brak{\vec{x}} = \vec{x}^\top\vec{V}\vec{x} + 2\vec{u}^\top\vec{x} + f = 0 
\end{align}
where
\begin{align}
	\vec{V} = \myvec{ 1 & 0 \\ 0 & 0},\, 
	\vec{u} = \myvec{0 \\ -2},\, 
	f = 0 
\end{align}
We are given that 
\begin{align}
	\vec{h} &= \myvec{4 \\ -2}
\end{align}
This can be formulated as optimization problem as below:
\begin{align}
	\label{eq:12/6/6/4/sdp/Eq3}
	&  \min_{\vec{x}} \quad \text{f}\brak{\vec{x}} = \norm{\vec{x}-\vec{h}}^2\\
	\label{eq:12/6/6/4/sdp/Eq4}
	& \text{s.t.}\quad g\brak{\vec{x}} = \vec{x}^\top\vec{V}\vec{x} + 2\vec{u}^\top\vec{x} + f = 0  
\end{align}
Now,
\begin{align}
	\norm{\vec{x}-\vec{h}}^2 &= \norm{\vec{x}}^2 - 2\vec{h}^\top\vec{x}+\norm{\vec{h}}^2\\
	&= \vec{y}^\top\vec{C}\vec{y}
\end{align}
where,
\begin{align}
	\vec{C} = \myvec{\vec{I}&-\vec{h}\\-\vec{h}^\top& \norm{h}^2} \text{ and }
	\vec{y} = \myvec{\vec{x}\\1}
\end{align}
And equation \eqref{eq:12/6/6/4/sdp/Eq4} can be expressed as
\begin{align}
	\vec{y}^\top\vec{A}\vec{y} = 0
\end{align}
where
\begin{align}
	\vec{A} = \myvec{\vec{V}&\vec{u}\\\vec{u}^\top & f}
\end{align}
Using SDR(Semi Definite Relaxation), \eqref{eq:12/6/6/4/sdp/Eq3} can be expressed as
\begin{align}
	& \min_{\vec{X}} tr\brak{\vec{C}\vec{X}}\\
	& \text{s.t. } tr\brak{\vec{A}\vec{X}} = 0\\
	& \vec{X} \succcurlyeq \vec{0}
\end{align}
On solving it yields to the point
\begin{align}
	\vec{x} = \myvec{1.695\\0.718}
\end{align}
This is same as we obtained in Gradient Descent and Lagrange multiplier. Hence this is the required point.

















