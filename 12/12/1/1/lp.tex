\iffalse
\documentclass[12pt]{article}
\usepackage{graphicx}
\usepackage[none]{hyphenat}
\usepackage{listings}
\usepackage[english]{babel}
\usepackage{caption} 
\usepackage{booktabs}
\usepackage{array}
\usepackage{amssymb} % for \because
\usepackage{extarrows} % for Row operations arrows
\usepackage{listings}
\lstset{
  frame=single,
  breaklines=true
}
\usepackage{hyperref}
\usepackage{mathtools}
%Following 2 lines were added to remove the blank page at the beginning
\usepackage{atbegshi}% http://ctan.org/pkg/atbegshi
\AtBeginDocument{\AtBeginShipoutNext{\AtBeginShipoutDiscard}}

%for Tables
\def\inputGnumericTable{}
\usepackage[latin1]{inputenc}
\usepackage{fullpage}
\usepackage{color}
\usepackage{array}
\usepackage{longtable}
\usepackage{calc}
\usepackage{multirow}
\usepackage{hhline}
\usepackage{ifthen}

%New macro definitions
\newcommand{\mydet}[1]{\ensuremath{\begin{vmatrix}#1\end{vmatrix}}}
\providecommand{\brak}[1]{\ensuremath{\left(#1\right)}}
\providecommand{\sbrak}[1]{\ensuremath{{}\left[#1\right]}}
\providecommand{\norm}[1]{\left\lVert#1\right\rVert}
\providecommand{\abs}[1]{\left\vert#1\right\vert}
\newcommand{\solution}{\noindent \textbf{Solution: }}
\newcommand{\myvec}[1]{\ensuremath{\begin{pmatrix}#1\end{pmatrix}}}
\let\vec\mathbf

\usepackage{amsmath}
\usepackage{graphicx}
%\usepackage[colorlinks=true, allcolors=blue]{hyperref}
\def\fnum@table{\tablename~\thetable}
\def\fnum@figure{\figurename~\thefigure}

\begin{document}

\begin{center}
\title{\textbf{Linear Programming}}
\date{\vspace{-5ex}} %Not to print date automatically
\maketitle
\end{center}
\setcounter{page}{1}

\section{12$^{th}$ Maths - Chapter 12}
This is Problem-1 from Exercise 12.1
\begin{enumerate}
\fi
\begin{enumerate}
\item Using cvxpy method: The given problem can be formulated as 
\begin{align}
	\max_{\vec{x}} Z &= \myvec{3 & 4}\vec{x} \\
        \myvec{1 & 4\\
               -1 &0\\
	       0 & -1}\vec{x}\preceq & \myvec{4 \\0\\0}
\end{align}
Solving using cvxpy, we get
\begin{align}
	\label{eq:12/12/1/1/maxval}
	\max_{\vec{x}} Z &= 12 , 
	\vec{x} = \myvec{4  \\  0} 
\end{align}
\item Using Corner point method: The corner points of  the inequalities are:
\begin{align}
	\vec{A} = \myvec{0 \\ 1}\\
	\vec{B} = \myvec{0 \\ 0} \\
	\vec{x} = \myvec{4 \\ 0} 
\end{align}
Substituting above values of corner points in Equation \eqref{eq:12/12/1/1/Obj_func} to get the value of $Z$, as shown in the Table \ref{tab:widgets}
\begin{table}[!h]
	\centering
	\input{12/12/1/1/tables/det.tex} 
	\caption{}
	\label{tab:widgets}
\end{table}

From the table \ref{tab:widgets}, it is clear that the optimum value and optimum point are similar to what we found in \eqref{eq:12/12/1/1/maxval}. 
\end{enumerate}
The relevant figure is as shown in \ref{fig:12/12/1/1/inequality}
\begin{figure}[!h]
	\begin{center}
		\includegraphics[width=\columnwidth]{12/12/1/1/figs/problem1.pdf}
	\end{center}
	\caption{}
	\label{fig:12/12/1/1/inequality}
\end{figure}
