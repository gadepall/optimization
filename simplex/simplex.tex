\documentclass[journal,12pt,twocolumn]{IEEEtran}
%
\usepackage{setspace}
\usepackage{gensymb}
\usepackage{xcolor}
\usepackage{caption}
\usepackage{chngcntr}
%\usepackage{subcaption}
%\doublespacing
\singlespacing
\usepackage{multicol}
%\usepackage{graphicx}
%\usepackage{amssymb}
%\usepackage{relsize}
\usepackage[cmex10]{amsmath}
\usepackage{mathtools}
%\usepackage{amsthm}
%\interdisplaylinepenalty=2500
%\savesymbol{iint}
%\usepackage{txfonts}
%\restoresymbol{TXF}{iint}
%\usepackage{wasysym}
\usepackage{amsthm}
\usepackage{mathrsfs}
\usepackage{txfonts}
\usepackage{stfloats}
\usepackage{cite}
\usepackage{cases}
\usepackage{subfig}
%\usepackage{xtab}
\usepackage{longtable}
\usepackage{multirow}
%\usepackage{algorithm}
%\usepackage{algpseudocode}
\usepackage{enumerate}
\usepackage{mathtools}
\usepackage{iithtlc}
%\usepackage[framemethod=tikz]{mdframed}
\usepackage{listings}
    \usepackage[latin1]{inputenc}                                 %%
    \usepackage{color}                                            %%
    \usepackage{array}                                            %%
    \usepackage{longtable}                                        %%
    \usepackage{calc}                                             %%
    \usepackage{multirow}                                         %%
    \usepackage{hhline}                                           %%
    \usepackage{ifthen}                                           %%
  %optionally (for landscape tables embedded in another document): %%
    \usepackage{lscape}     

%\usepackage{stmaryrd}


%\usepackage{wasysym}
%\newcounter{MYtempeqncnt}
\DeclareMathOperator*{\Res}{Res}
%\renewcommand{\baselinestretch}{2}
\renewcommand\thesection{\arabic{section}}
\renewcommand\thesubsection{\thesection.\arabic{subsection}}
\renewcommand\thesubsubsection{\thesubsection.\arabic{subsubsection}}

\renewcommand\thesectiondis{\arabic{section}}
\renewcommand\thesubsectiondis{\thesectiondis.\arabic{subsection}}
\renewcommand\thesubsubsectiondis{\thesubsectiondis.\arabic{subsubsection}}

% correct bad hyphenation here
\hyphenation{op-tical net-works semi-conduc-tor}

\def\inputGnumericTable{}  

\lstset{
language=python,
frame=single, 
breaklines=true
}
\newcommand\bigzero{\makebox(0,0){\text{\huge0}}}
%\lstset{
	%%basicstyle=\small\ttfamily\bfseries,
	%%numberstyle=\small\ttfamily,
	%language=Octave,
	%backgroundcolor=\color{white},
	%%frame=single,
	%%keywordstyle=\bfseries,
	%%breaklines=true,
	%%showstringspaces=false,
	%%xleftmargin=-10mm,
	%%aboveskip=-1mm,
	%%belowskip=0mm
%}

%\surroundwithmdframed[width=\columnwidth]{lstlisting}


\begin{document}
%

\theoremstyle{definition}
\newtheorem{theorem}{Theorem}[section]
\newtheorem{problem}{Problem}
\newtheorem{proposition}{Proposition}[section]
\newtheorem{lemma}{Lemma}[section]
\newtheorem{corollary}[theorem]{Corollary}
\newtheorem{example}{Example}[section]
\newtheorem{definition}{Definition}[section]
%\newtheorem{algorithm}{Algorithm}[section]
%\newtheorem{cor}{Corollary}
\newcommand{\BEQA}{\begin{eqnarray}}
\newcommand{\EEQA}{\end{eqnarray}}
\newcommand{\define}{\stackrel{\triangle}{=}}

\bibliographystyle{IEEEtran}
%\bibliographystyle{ieeetr}

\providecommand{\nCr}[2]{\,^{#1}C_{#2}} % nCr
\providecommand{\nPr}[2]{\,^{#1}P_{#2}} % nPr
\providecommand{\mbf}{\mathbf}
\providecommand{\pr}[1]{\ensuremath{\Pr\left(#1\right)}}
\providecommand{\qfunc}[1]{\ensuremath{Q\left(#1\right)}}
\providecommand{\sbrak}[1]{\ensuremath{{}\left[#1\right]}}
\providecommand{\lsbrak}[1]{\ensuremath{{}\left[#1\right.}}
\providecommand{\rsbrak}[1]{\ensuremath{{}\left.#1\right]}}
\providecommand{\brak}[1]{\ensuremath{\left(#1\right)}}
\providecommand{\lbrak}[1]{\ensuremath{\left(#1\right.}}
\providecommand{\rbrak}[1]{\ensuremath{\left.#1\right)}}
\providecommand{\cbrak}[1]{\ensuremath{\left\{#1\right\}}}
\providecommand{\lcbrak}[1]{\ensuremath{\left\{#1\right.}}
\providecommand{\rcbrak}[1]{\ensuremath{\left.#1\right\}}}
\theoremstyle{remark}
\newtheorem{rem}{Remark}
\newcommand{\sgn}{\mathop{\mathrm{sgn}}}
\providecommand{\abs}[1]{\left\vert#1\right\vert}
\providecommand{\res}[1]{\Res\displaylimits_{#1}} 
\providecommand{\norm}[1]{\lVert#1\rVert}
\providecommand{\mtx}[1]{\mathbf{#1}}
\providecommand{\mean}[1]{E\left[ #1 \right]}
\providecommand{\fourier}{\overset{\mathcal{F}}{ \rightleftharpoons}}
%\providecommand{\hilbert}{\overset{\mathcal{H}}{ \rightleftharpoons}}
\providecommand{\system}{\overset{\mathcal{H}}{ \longleftrightarrow}}
	%\newcommand{\solution}[2]{\textbf{Solution:}{#1}}
\newcommand{\solution}{\noindent \textbf{Solution: }}
\providecommand{\dec}[2]{\ensuremath{\overset{#1}{\underset{#2}{\gtrless}}}}
%\numberwithin{equation}{subsection}
\numberwithin{equation}{section}
%\numberwithin{equation}{problem}
%\numberwithin{problem}{subsection}
\numberwithin{problem}{section}
\counterwithin{table}{problem}
%\numberwithin{definition}{subsection}
\makeatletter
\@addtoreset{figure}{problem}
\makeatother
%\makeatletter
%\@addtoreset{table}{problem}
%\makeatother

\let\StandardTheFigure\thefigure
%\let\StandardTheTable\thetable
%\renewcommand{\thefigure}{\theproblem.\arabic{figure}}
\renewcommand{\thefigure}{\theproblem}
%\renewcommand{\thetable}{\theproblem}
%\numberwithin{figure}{section}

%\numberwithin{figure}{subsection}

\def\putbox#1#2#3{\makebox[0in][l]{\makebox[#1][l]{}\raisebox{\baselineskip}[0in][0in]{\raisebox{#2}[0in][0in]{#3}}}}
     \def\rightbox#1{\makebox[0in][r]{#1}}
     \def\centbox#1{\makebox[0in]{#1}}
     \def\topbox#1{\raisebox{-\baselineskip}[0in][0in]{#1}}
     \def\midbox#1{\raisebox{-0.5\baselineskip}[0in][0in]{#1}}

%\vspace{3cm}

\title{
\logo{
Simplex Method
}
}
%\centering \textbf{\Large Optimization}\\
%\bigskip
\author{Y Aditya and G V V Sharma$^{*}$% <-this % stops a space
\thanks{* The authors are with the Department
of Electrical Engineering, Indian Institute of Technology, Hyderabad
502285 India e-mail:  gadepall@iith.ac.in.}% <-this % stops a space
%\thanks{J. Doe and J. Doe are with Anonymous University.}% <-this % stops a space
%\thanks{Manuscript received April 19, 2005; revised January 11, 2007.}}
}
%
\maketitle
%
\begin{abstract}
This manual explains the Simplex Method for solving Linear Programming problems through examples.
\end{abstract}
%
\section{Iterations and Pivoting}
%\textit{Finding optimal values of linear expressions w.r.t linear constraints is called linear programming}\\
\begin{problem}
Maximize
\begin{align}
f = 6x_1 + 5x_2 
\end{align}
with constraints
\begin{align}
x_1 + x_2 &\leq 5 \nonumber \\
3x_1 + 2x_2 &\leq 12 \nonumber \\
\text{ where } x_1,x_2 &\geq 0 \nonumber
\end{align}
\end{problem}
\solution
Introduce two dummy variables $x_3,x_4$ to convert inequalities to equations 
\begin{align}
\label{eq:slack}
\begin{split}
x_1 + x_2 + x_3 &= 5 \\
3x_1 + 2x_2 + x_4 &= 12 \\ 
\text{ where } x_3,x_4 &\geq 0 
\end{split}
\end{align}
\begin{enumerate}[1.]
\item 
From \eqref{eq:slack}, we obtain
\begin{align}
\label{eq:x3}
x_3 &= 5 - x_1 - x_2\\
x_4 &= 12 - 3x_1 - 2x_2 
\label{xonesmall} 
\end{align}
Setting $x_2 = 0$ in \eqref{xonesmall},
\begin{align}
x_3 > 0 \Rightarrow 5-x_1 &> 0 \Rightarrow x_1 < 5  \nonumber\\
x_4 > 0 \Rightarrow 12-3x_1 &> 0 \Rightarrow x_1 < 4 
\end{align}
and
\begin{equation}
f_1 = 6x_1 + 5x_2 
\end{equation}
\item (Pivoting):
\eqref{xonesmall} results in a lower bound for $x_1$. Rearranging \eqref{xonesmall},
\begin{align}
x_1 &= 4 - \frac{2}{3}x_2 - \frac{1}{3}x_4\\
x_3 &= 5 - x_1 - x_2 \nonumber\\
&= 5 - \brak{4 - \frac{2}{3}x_2 - \frac{1}{3}x_4} - x_2 \nonumber\\
 &= 1 - \frac{1}{3}x_2 + \frac{1}{3}x_4
 \label{eq:x3_1}
 \end{align}
and substituting $x_4 = 0$ results in
 \begin{align}
x_1 > 0 &\implies x_2 < 6 \\
x_3 > 0 &\implies x_2 < 3 
\end{align}
and
\begin{align}
f_2 &= 6\brak{4 - \frac{2}{3}x_2 - \frac{1}{3}x_4} + 5x_2\\
&= 24 + x_2 - 2x_4
\end{align}

\item The lower bound  $x_2 < 3$ results from \eqref{eq:x3_1}.  
Choosing \eqref{eq:x3_1} for pivoting for $x_2$ and rearranging,
%
\begin{align}
x_2 &= 1 + x_4 - 3x_3,
\\
x_1 &= 4 - \frac{2}{3}\brak{1 + x_4 - 3x_3} - \frac{1}{3}x_4\\
&= \frac{10}{3} - x_4 +2x_3
\end{align}
%
and
\begin{align}
f_3 &= 24 + x_2 - 2x_4
\\
&= 24 + \brak{1 + x_4 - 3x_3} - 2x_4
\\
&= 25 - x_4 - 3x_3
\end{align}
\end{enumerate}
Since $x_3,x_4 \ge 0$, the maximum value of $f_3 = 25 $ and this is the desired answer.  The iteration ends when the coefficients of the variables in $f_i$, where $i$ is the $i$th iteration are
all negative. 
\begin{enumerate}
\item[5)] Complete iteration three, pivoting $x_2$, and find the maximum value of the expression. Solve for corresponding  $x_1$ and $x_2$
\end{enumerate}
\begin{problem}
Maximise $5x_1 + 3x_2$ w.r.t the constraints
\begin{align}
x_1 + x_2 &\leq 2 \nonumber\\
5x_1 + 2x_2 &\leq 10 \nonumber\\
3x_1 + 8x_2 &\leq 12 \nonumber\\
\text{ where } x_1,x_2 &\geq 0 \nonumber
\end{align}
\end{problem}
\section{Tabular Method}
\begin{problem}
Maximize
\begin{align}
f = 6x_1 + 8x_2
\label{eq:f_tabular}
\end{align}
with constraints:
\begin{align}
x_1 + x_2 &\leq 10\\
2x_1 + 3x_2 &\leq 25\\
x_1 + 5x_2 &\leq 35
\end{align}
\end{problem}
\solution
Introducing dummy variables $x_3,x_4,x_5$
\begin{align}
x_1 + x_2 + x_3 &= 10\\
2x_1 + 3x_2 + x_4 &=25\\
x_1 + 5x_2 + x_5 &= 35
\end{align}
The objective function in \eqref{eq:f_tabular}  becomes
\begin{align}
\label{eq:f_x1-5}
f = 6x_1 + 8x_2 + 0x_3 + 0x_4 + 0x_5
\end{align}
\begin{enumerate}[1.]
\item Set up a table as below.
\begin{table}[!h]
\begin{center}
\begin{tabular}{l  l | l l l l l | l  }
                 & & 6 & 8 & 0 & 0 & 0 &  \\
                 & & $x_1$ & $x_2$ & $x_3$ & $x_4$ & $x_5$  & RHS  \\
\hline
0 & $x_3$ & 1 & 1 & 1 & 0 & 0 & 10   \\
0 & $x_4$ & 2 & 3 & 0 & 1 & 0 & 25  \\
0 & $x_5$ & 1 & 5 & 0 & 0 & 1 & 35 \\
\end{tabular}
\end{center}
\caption{}
\end{table}
Note that the numbers in the first row are the coefficents of $x_1,x_2$ in \eqref{eq:f_tabular}.
\item Define, $c_j$ $=$ coefficient of $x_j$ in \eqref{eq:f_x1-5}, where $j$ $=$ $1,2,3,4,5$.
\begin{align}
z_1 &= 
\begin{pmatrix}
0
\\
0
\\
0
\end{pmatrix}
.
\begin{pmatrix}
1
\\
2
\\
1
\end{pmatrix}
\\
z_2 &= 
\begin{pmatrix}
0
\\
0
\\
0
\end{pmatrix}
.
\begin{pmatrix}
1
\\
3
\\
5
\end{pmatrix}
\\
\vdots \nonumber
\\
z_{RHS} &= 
\begin{pmatrix}
0
\\
0
\\
0
\end{pmatrix}
.
\begin{pmatrix}
10
\\
25
\\
35
\end{pmatrix}
\end{align}
%
where the scalar products of the column vectors are being computed.
%
\begin{table}[!h]
\begin{center}
\begin{tabular}{l  l | l l l l l | l }
                 & & 6 & 8 & 0 & 0 & 0 & \\
                 & & $x_1$ & $x_2$ & $x_3$ & $x_4$ & $x_5$  & RHS \\
\hline
0 & $x_3$ & 1 & 1 & 1 & 0 & 0 & 10   \\
0 & $x_4$ & 2 & 3 & 0 & 1 & 0 & 25 \\
0 & $x_5$ & 1 & 5 & 0 & 0 & 1 & 35\\
\hline
&$c_j-z_j$ & 6 &  \textbf{8}$\uparrow$ & 0 & 0 & 0 & $z_{RHS}=$0 \\
\end{tabular}
\end{center}
\caption{}
\label{table:x2_enter}
\end{table}
%
\item 
For $j=2, c_j-z_j = 8-0 = 8 $ which is the largest among all $j$. So $x_2$ is going to enter into next iteration.
Which variable is going to leave?
Make a new column $\theta$ as in Table \ref{table:x5_leave}  where
\begin{align}
\theta = \frac{\text{RHS value}}{\text{corresponding values in column of }x_2}
\end{align} 
\begin{table}[!h]
\begin{center}
\resizebox{\columnwidth}{!}{%
\begin{tabular}{l  l | l l l l l | l | l }
                 & & 6 & 8 & 0 & 0 & 0 & & \\
                 & & $x_1$ & $x_2$ & $x_3$ & $x_4$ & $x_5$  & RHS & $\theta$ \\
\hline
0 & $x_3$ & 1 & 1 & 1 & 0 & 0 & 10 & 10  \\
0 & $x_4$ & 2 & 3 & 0 & 1 & 0 & 25 & $\frac{25}{3}$ \\
0 & $x_5$ & 1 & \textbf{5} & 0 & 0 & 1 & 35 & \textbf{7}$\longrightarrow$ \\
\hline
&$c_j-z_j$ & 6 & \textbf{8}$\uparrow$ & 0 & 0 & 0 & $z_{RHS}=$0 & \\
\end{tabular}
}
\end{center}
\caption{}
\label{table:x5_leave}
\end{table}
\item From Table \ref{table:x5_leave}, it is clear that   7 is the smallest value of $\theta$ and corresponds to the row with $x_5$.  So $x_5$ is going to leave
the iteration. 5 in Table \ref{table:x5_leave} corresponds to the $x_2$ column (max $c_j - z_j$) and $x_5$ row (min $\theta$) is called the \textbf{pivot}.
The pivot element should be made $1$ in the next iteration
\begin{align}
Row_{x_5} = \frac{Row_{x_5}}{5}
\end{align}
\item As $x_5$ is leaving, this becomes row corresponding to $x_2$. 
All the other elements of column containing pivot should be made $0$ by row operations. See Table \ref{table:x2-x5}.
\begin{align}
R_{x_3} = R_{x_3} - R_{x_2}\\
R_{x_4} = R_{x_4} - 3R_{x_2}\\
\end{align}
\begin{table}[!h]
\begin{center}
\resizebox{\columnwidth}{!}{%
\begin{tabular}{l  l | l l l l l | l | l }
                 & & 6 & 8 & 0 & 0 & 0 & & \\
                 & & $x_1$ & $x_2$ & $x_3$ & $x_4$ & $x_5$  & RHS & $\theta$ \\
\hline
0 & $x_3$ & 1 & 1 & 1 & 0 & 0 & 10 & 10  \\
0 & $x_4$ & 2 & 3 & 0 & 1 & 0 & 25 & $\frac{25}{3}$ \\
0 & $x_5$ & 1 & \textbf{5} & 0 & 0 & 1 & 35 & \textbf{7}$\longrightarrow$ \\
\hline
&$c_j-z_j$ & 6 & \textbf{8}$\uparrow$ & 0 & 0 & 0 & $z_{RHS}=$0 & \\
\hline
0 & $x_3$ & 4/5 & 0 & 1 & 0 & -1/5 & 3 & 15/4  \\
0 & $x_4$ & \textbf{2/5}  & 0 & 0 & 1 & -3/5 & 4 & \textbf{20/7} $\longrightarrow$ \\
8 & $x_2$ & 1/5 & 1 & 0 & 0 & 1/5 & 7 & 35 \\
\hline
&$c_j-z_j$ & \textbf{22/5}$\uparrow$ & 0 & 0 & 0 & -8/5 & $z_{RHS}=$56 & \\
\end{tabular}
}
\end{center}
\caption{}
\label{table:x2-x5}
\end{table}
\item Continue the table until all $\mathbf{c_j-z_j}$ values are either zero or negative. The corresponding final $\mathbf{z_{RHS}}$ is the maximum value. Also find corresponding $\mathbf{x_1}$ and $\mathbf{x_2}$

\end{enumerate}

NOTE:
\begin{enumerate}
\item $\theta$ is always positive. In case it is coming as negative or undefined, leave the slot in the table blank.
\item Don't convert fractions into decimals. Final answers are natural numbers.
\end{enumerate}
\begin{problem}
Maximise
\begin{align}
6x_1 + 5x_2 \nonumber
\end{align}
with the constraints 
\begin{align}
x_1 + x_2 &\leq 5 \nonumber \\
3x_1 + 2x_2 &\leq 12 \nonumber \\
\text{ where } x_1,x_2 &\geq 0 \nonumber
\end{align}
using the tabular method. Find the corresponding values of $x_1$ \& $x_2$. 

\end{problem}

\end{document}

