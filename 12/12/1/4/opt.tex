\documentclass[a4paper,12pt,twocolumn]{article}
\usepackage{graphicx}
\usepackage[margin=0.5in]{geometry}
\usepackage[cmex10]{amsmath}
\usepackage{array}
\usepackage{gensymb}
\usepackage{booktabs}
\title{Optimization Assignment}

\author{Manideep Parusha- FWC22004}
\date{September 2022}
\providecommand{\norm}[1]{\left\lVert#1\right\rVert}
\providecommand{\abs}[1]{\left\vert#1\right\vert}
\let\vec\mathbf
\newcommand{\myvec}[1]{\ensuremath{\begin{pmatrix}#1\end{pmatrix}}}
\newcommand{\mydet}[1]{\ensuremath{\begin{vmatrix}#1\end{vmatrix}}}
\providecommand{\brak}[1]{\ensuremath{\left((#1\right)}}
\begin{document}
\maketitle
\section{Problem:}
Minimize $Z$ = $3x+5y$ such that $x+3y\ge3$, $x+y\ge2$, $x\ge0$, $y\ge0$.
\maketitle
\section{Solution:}
%\subsection{Theory:}
We need to first graph the feasible region of the system of inequalities. The feasible region is shown in the above figure in color red. The region is bounded. We need to find the coordinates of corner points and figure the minimum value of $Z$. The given set of equations are:
\begin{align}
	x+3y\ge3\\
	x+y\ge2\\
	x\ge0\\
	y\ge0
\end{align}
In vector form, they are written as:
\begin{align}
	&\myvec{1 & 3\\1&1}\myvec{x\\y}\ge\myvec{3\\2}\\
	&\myvec{1&0\\0&1}\myvec{x\\y}\ge\myvec{0\\0}
\end{align}
%\subsection{Mathematical Calculation:}
We need to find the intersection of given system of inequalities, to figure out the coordinates of feasible region.
The coordinates of the quadrilateral are $\myvec{0\\2}$, $\myvec{\frac{3}{2}\\\frac{1}{2}}$, $\myvec{3\\0}$.
Solving using cvxpy, we get,
\begin{align}
	&\myvec{x\\y} = \myvec{\frac{3}{2}\\\frac{1}{2}}\\
	&Z_{min} = 7\\
\end{align}
Hence, the minimum of Z is verified using optimization.
 
%\section{Construction:}
%The construction of system of equations can be done by plotting them using matplotlib library.
%\begin{table}[h]
 %       \centering
%\setlength\extrarowheight{2pt}
 %       \begin{tabular}{|c|c|c|}
  %              \hline
%		\textbf{variable} & \textbf{equation} & \textbf{comments}\\
%		\hline
%		y1 & (3-x)/3 & eqn 1\\
%		\hline
%		y2 & (2-x) & eq 2\\
%		\hline
%	\end{tabular}
%\end{table}

\end{document}
