\iffalse
\documentclass[journal,12pt,twocolumn]{IEEEtran}
\usepackage{romannum}
\usepackage{float}
\usepackage{setspace}
\usepackage{gensymb}
\singlespacing
\usepackage[cmex10]{amsmath}
\usepackage{amsthm}
\usepackage{mathrsfs}
\usepackage{txfonts}
\usepackage{stfloats}
\usepackage{bm}
\usepackage{cite}
\usepackage{cases}
\usepackage{subfig}
\usepackage{longtable}
\usepackage{multirow}
\usepackage{enumitem}
\usepackage{mathtools}
\usepackage{steinmetz}
\usepackage{tikz}
\usepackage{circuitikz}
\usepackage{verbatim}
\usepackage{tfrupee}
\usepackage[breaklinks=true]{hyperref}
\usepackage{tkz-euclide}
\usetikzlibrary{calc,math}
\usepackage{listings}
    \usepackage{color}                                            %%
    \usepackage{array}                                            %%
    \usepackage{longtable}                                        %%
    \usepackage{calc}                                             %%
    \usepackage{multirow}                                         %%
    \usepackage{hhline}                                           %%
    \usepackage{ifthen}                                           %%
  %optionally (for landscape 12/6/5/17/tables/ embedded in another document): %%
    \usepackage{lscape}     
\usepackage{multicol}
\usepackage{chngcntr}
\DeclareMathOperator*{\Res}{Res}
\renewcommand\thesection{\arabic{section}}
\renewcommand\thesubsection{\thesection.\arabic{subsection}}
\renewcommand\thesubsubsection{\thesubsection.\arabic{subsubsection}}

\renewcommand\thesectiondis{\arabic{section}}
\renewcommand\thesubsectiondis{\thesectiondis.\arabic{subsection}}
\renewcommand\thesubsubsectiondis{\thesubsectiondis.\arabic{subsubsection}}

% correct bad hyphenation here
\hyphenation{op-tical net-works semi-conduc-tor}
\def\inputGnumericTable{}                                 %%

\lstset{
frame=single, 
breaklines=true,
columns=fullflexible
}

\begin{document}


\newtheorem{theorem}{Theorem}[section]
\newtheorem{problem}{Problem}
\newtheorem{proposition}{Proposition}[section]
\newtheorem{lemma}{Lemma}[section]
\newtheorem{corollary}[theorem]{Corollary}
\newtheorem{example}{Example}[section]
\newtheorem{definition}[problem]{Definition}
\newcommand{\BEQA}{\begin{eqnarray}}
\newcommand{\EEQA}{\end{eqnarray}}
\newcommand{\define}{\stackrel{\triangle}{=}}

\bibliographystyle{IEEEtran}
\providecommand{\mbf}{\mathbf}
\providecommand{\pr}[1]{\ensuremath{\Pr\left(#1\right)}}
\providecommand{\qfunc}[1]{\ensuremath{Q\left(#1\right)}}
\providecommand{\sbrak}[1]{\ensuremath{{}\left[#1\right]}}
\providecommand{\lsbrak}[1]{\ensuremath{{}\left[#1\right.}}
\providecommand{\rsbrak}[1]{\ensuremath{{}\left.#1\right]}}
\providecommand{\brak}[1]{\ensuremath{\left(#1\right)}}
\providecommand{\lbrak}[1]{\ensuremath{\left(#1\right.}}
\providecommand{\rbrak}[1]{\ensuremath{\left.#1\right)}}
\providecommand{\cbrak}[1]{\ensuremath{\left\{#1\right\}}}
\providecommand{\lcbrak}[1]{\ensuremath{\left\{#1\right.}}
\providecommand{\rcbrak}[1]{\ensuremath{\left.#1\right\}}}
\theoremstyle{remark}
\newtheorem{rem}{Remark}
\newcommand{\sgn}{\mathop{\mathrm{sgn}}}
\providecommand{\abs}[1]{\left\vert#1\right\vert}
\providecommand{\res}[1]{\Res\displaylimits_{#1}} 
\providecommand{\norm}[1]{\left\lVert#1\right\rVert}
\providecommand{\mtx}[1]{\mathbf{#1}}
\providecommand{\mean}[1]{E\left[ #1 \right]}
\providecommand{\fourier}{\overset{\mathcal{F}}{ \rightleftharpoons}}
\providecommand{\system}{\overset{\mathcal{H}}{ \longleftrightarrow}}
\newcommand{\solution}{\noindent \textbf{Solution: }}
\newcommand{\cosec}{\,\text{cosec}\,}
\providecommand{\dec}[2]{\ensuremath{\overset{#1}{\underset{#2}{\gtrless}}}}
\newcommand{\myvec}[1]{\ensuremath{\begin{pmatrix}#1\end{pmatrix}}}
\newcommand{\mydet}[1]{\ensuremath{\begin{vmatrix}#1\end{vmatrix}}}
\numberwithin{equation}{subsection}
\makeatletter
\@addtoreset{figure}{problem}
\makeatother

\let\StandardTheFigure\thefigure
\let\vec\mathbf
\renewcommand{\thefigure}{\theproblem}



\def\putbox#1#2#3{\makebox[0in][l]{\makebox[#1][l]{}\raisebox{\baselineskip}[0in][0in]{\raisebox{#2}[0in][0in]{#3}}}}
     \def\rightbox#1{\makebox[0in][r]{#1}}
     \def\centbox#1{\makebox[0in]{#1}}
     \def\topbox#1{\raisebox{-\baselineskip}[0in][0in]{#1}}
     \def\midbox#1{\raisebox{-0.5\baselineskip}[0in][0in]{#1}}

\vspace{3cm}


\title{Assignment 5}
\author{Jaswanth Chowdary Madala}


% make the title area
\maketitle

\newpage

%\tableofcontents

\bigskip

\renewcommand{\thefigure}{\theenumi}
\renewcommand{\thetable}{\theenumi}

\begin{enumerate}
\textbf{Solution:} 
	\fi
		Let the given side of tin be $a$.
\begin{align}
a = 18cm
\end{align}
Lets cut a square of side $x$ from each corner then the box formed folding up the flaps has dimensions as
\begin{align}
l = a-2x, \, b = a-2x, \, h = x
\end{align}
The length, breadth, height are positive. These give the constraints on $x$
\begin{align}
a-2x > 0,\,  x > 0\\
\implies 0 < x < \frac{a}{2} \label{eq:12/6/5/17/1}
\end{align} 
Volume of the box is given by
\begin{align}
V\brak{x} = x(a-2x)^2
\end{align}
\begin{enumerate}
\item We now check the convexity of the function $V\brak{x}$ under the constraints given by \eqref{eq:12/6/5/17/1}
\begin{align}
V^{\prime}\brak{x} &= (a-2x)(a-6x)\\
V^{\prime\prime}\brak{x} &= 8(3x - a)
\end{align}
For $x > \frac{a}{3}$ the function $V\brak{x}$ is convex. The function convexity is changing under the constraints. So the given problem cannot be expressed as a Convex optimization problem. 
\item The problem can be solved using the Gradient Descent Algorithm. It can be modified as
\begin{align}
\min_{x}f(x) = -V\brak{x} \label{eq:12/6/5/17/2}
\end{align}
Here we have,
\begin{align}
f(x) &= -x(a-2x)^2 \label{eq:12/6/5/17/3}\\
f^{\prime}\brak{x} &= -(a-2x)(a-6x) \label{eq:12/6/5/17/4}
\end{align}
A numerical solution for \eqref{eq:12/6/5/17/2} is obtained as
\begin{align}
\lambda_{n+1} &= \lambda_{n} - \alpha f^{\prime}\brak{x}
\end{align}
where $\lambda_0$ is an inital guess and $\mu$ is a variable parameter. These parameters decide how fast the algorithm converges.
By taking the parameters as listed in the below table
\begin{table}[h]
\centering
%%%%%%%%%%%%%%%%%%%%%%%%%%%%%%%%%%%%%%%%%%%%%%%%%%%%%%%%%%%%%%%%%%%%%%
%%                                                                  %%
%%  This is a LaTeX2e table fragment exported from Gnumeric.        %%
%%                                                                  %%
%%%%%%%%%%%%%%%%%%%%%%%%%%%%%%%%%%%%%%%%%%%%%%%%%%%%%%%%%%%%%%%%%%%%%%

\begin{center}
\begin{tabular}{|c|c|c|}
\hline
\textbf{Parameter}& \textbf{Description} &\textbf{Value}\\ \hline
$\lambda_0$		 &	Initial guess& 8.5 \\ \hline
$\alpha $		 &	Variable parameter & 0.01\\ \hline
$N$		 	     &  Number of iterations &   10000\\ \hline
$\epsilon$		 &  Tolerance in $\lambda$	& $10^{-6}$ \\ \hline
\end{tabular}
\end{center}

\caption{}
\label{tab:12/6/5/17/1}
\end{table}

The value of $x$ obtained is
\begin{align}
x = 3
\end{align}
From equation \eqref{eq:12/6/5/17/2}, the minimum value of $f\brak{x}$ is equivalent to maximum value of $V\brak{x}$. Hence the maximum value of $V\brak{x}$ occurs at $x = 3$.
\begin{align}
V\brak{3} = 432 
\end{align}
\end{enumerate}
