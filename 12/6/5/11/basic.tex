\documentclass[10pt,twocolumn]{article}
\usepackage{graphicx}
\usepackage[margin=0.5in]{geometry}
\usepackage[cmex10]{amsmath}
\usepackage{array}
\usepackage{booktabs}
\usepackage{mathtools}
\title{\textbf{Optimization}}
\author{Vemulapalli Bavya Sri}
\date{October 2022}


\providecommand{\norm}[1]{\lVert#1\rVert}
\providecommand{\abs}[1]{\vert#1\vert}
\let\vec\mathbf
\newcommand{\myvec}[1]{\ensuremath{\begin{pmatrix}#1\end{pmatrix}}}
\newcommand{\mydet}[1]{\ensuremath{\begin{vmatrix}#1\end{vmatrix}}}
\providecommand{\brak}[1]{\ensuremath{\left(#1\right)}}
\providecommand{\lbrak}[1]{\ensuremath{\left(#1\right.}}
\providecommand{\rbrak}[1]{\ensuremath{\left.#1\right)}}
\providecommand{\sbrak}[1]{\ensuremath{{}\left[#1\right]}}

\begin{document}

\maketitle
\paragraph{\textit{Problem Statement} - It is given that at x=1, the function
$x^4-62x^2+ax+9$ attains its maximum value, on the interval [0,2]. Find the value of a.} 
\section{Solution}
\begin{flushleft}
Given function is,\\
\end{flushleft}
\begin{equation}
\label{eqn:1}
    f(x)=x^4-62x^2+ax+9
\end{equation}
\subsection{Calculation of Maxima using normal differentiation}
\begin{flushleft}
Differentiating above Eq(1), we get,
\end{flushleft}
\center
$\nabla f(x) = 4x^3-124x+a$
\endcenter
\begin{flushleft}
f attains its maximum value on the interval [0,2] at x=1.
\end{flushleft}
\center
$\implies \nabla f(1) =0$\endcenter
\center
$\implies a=120$\endcenter

\begin{flushleft}
\subsection{Calculation of Maxima using gradient ascent algorithm}
\end{flushleft}
\begin{flushleft}
Maxima of the above equation (1), can be calculated from the following expression,\\
To find,
\end{flushleft}
\begin{align}
\max_{x} f(x)
\end{align}  
    \begin{equation}
        x_{n+1}= x_n + \alpha \nabla f(x_n)
    \end{equation}
\begin{flushleft}
Taking $x_0=0.5,\alpha=0.001$ and precision = 0.00000001, values obtained using python are:
\end{flushleft} 
\center
        \boxed{$\text{Maxima} = 68$}\\
        \vspace{0.45cm}
        \boxed{$\text{Maxima Point} = 1$}\\
        \vspace{0.45cm}
        \boxed{$\text{a} = 120$}
\endcenter

\begin{flushleft}
\section{Construction}
\end{flushleft}

\begin{flushleft}
1. At first, the given function has been differentiated and it is solved by setting f'(x) equal to zero. By using x values, f(x) values are calculated.\\
\vspace{0.25cm}
2. Later, the given function f(x) is solved by gradient ascent algorithm to find maxima and the point at which f(x) is maximum.\\
\vspace{0.25cm}
3. Maxima and related points are, \\
\vspace{0.25cm}
\center
Maxima point, Max=(1 , 68) 
\end{flushleft}

\begin{figure}[h]
\includegraphics[scale=0.6]{b.png}
\caption{Graph}
\label{fig:Graph}
\end{figure}

\end{document}