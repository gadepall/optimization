\iffalse
\documentclass[journal,10pt,twocolumn]{article}
\usepackage{graphicx}
\usepackage[margin=0.5in]{geometry}
\usepackage{amsmath}
\usepackage{array}
\usepackage{booktabs}
\usepackage{listings}
\providecommand{\norm}[1]{\left\lVert#1\right\rVert}
\providecommand{\abs}[1]{\left\vert#1\right\vert}
\usepackage{enumerate}
\let\vec\mathbf
\newcommand{\myvec}[1]{\ensuremath{\myvec{#1}}}
\newcommand{\mydet}[1]{\ensuremath{\begin{vmatrix}#1\end{vmatrix}}}
\providecommand{\brak}[1]{\ensuremath{\left(#1\right)}}
\lstset{
frame=single,
breaklines=true,
columns=fullflexible
}
\title{\textbf{Matrix Assignment}}
\author{Mannava Venkatasai}
\date{September 2022}
\begin{document}
\maketitle
\raggedright \textbf{Problem Statement}: \vspace{3mm} \\
\fi
Two godowns A and B have grain capacity of 100 quintals and 50 quintals
respectively. They supply to 3 ration shops, D, E and F whose requirements are
60, 50 and 40 quintals respectively. The cost of transportation per quintal from
the godowns to the shops are given in the following table
\begin{table}[!ht]
	\centering
\begin{tabular}{|c|c|c|}
\hline
% \begin{tabularx}{\linewidth} {lX}
 From/to & A& B  \\ 
 \hline
 D & 6 & 4 \\  
 \hline
 E & 3  & 2 \\
 \hline
  F & 2.5 & 3 \\
 \hline
\end{tabular} 
\end{table} 
\vspace{5mm}
How should the supplies be transported in order that the transportation cost is minimum? What is the minimum cost?
\\
\solution
Let's assume that 
\begin{enumerate}
\item A supplies $x$ quintals grain to ration shop D.
\item A supplies $y$ quintals grain to ration shop E.
\item A will supply remaining grains 100-$x$-$y$ quintals to F.
\item B will supply 60-$x$ quintals grain to ration shop D. 
\item B will supply 50-$y$ quintals grain to ration shop E.
\item B will supply $x$+$y$-60 quintals grain to ration shop F.
\end{enumerate}
Total transportation cost is given by :
\begin{align}
P=2.5x+1.5y+410
\end{align}
Now, Since godown A can supply maximum 60 quintals to ration shop D and 50 quintals to ration shop E and have maximum 100 quintals capacity to supply.\vspace{2mm} \\ Also, if godown A supplies all 40 quintals to ration shop F, then remaining 60 quintals will be supplied to ration shop D and E and $x$ and $y$ is amount of grains. It can never be negative.  This leads to the following conditions
\begin{align}
x+y \le 100 \\
x \le 60 \\
y \le 50 \\
-x-y \le -60 \\
x \ge 0 \\
y \ge 0
\end{align}
\iffalse
The above equations in vector form is :
\begin{align}
\vec{A_1} = 
\myvec{
1 \\
1 \\
} \\
\vec{A_2} = 
\myvec{
1 \\
1 \\
} \\
\vec{A_3} = 
\myvec{
1 \\
0 \\
} \\
\vec{A_4} = 
\myvec{
0 \\
1 \\
} \\
\vec{x} = 
\myvec{
x \\
y \\
}
\end{align}
\begin{align}
\vec{A_1}\vec{x} \le 100
\end{align}
\begin{align}
\vec{A_2}\vec{x} \ge 60
\end{align}
\begin{align}
\vec{A_3}\vec{x} \le 60
\end{align}
\begin{align}
\vec{A_4}\vec{x} \le 50
\end{align}
which can be expressed in vector form as
\fi
The optimization problem can then be expressed as
\begin{align}
	P=\max_{\vec{x}}\myvec{2.5 & 1.5}\vec{x}+410
	\\
	s.t. \quad
 \myvec{1 &1 \\ -1 & -1 \\ -1 & 0 \\ 0 & -1 \\} \vec{x}\preceq \myvec{100 \\ -60 \\ -60 \\ -50}
\end{align}
yielding
\begin{align}
	P = 510, 
\vec{x} = 
\myvec{
10 \\
50 \\
}
\end{align}
Hence,
\begin{enumerate}
\item The minimum transportation cost is : 510 /-
\item A supplies 10 quintals grain to ration shop D.
\item A supplies 50 quintals grain to ration shop E.
\item A supplies 40 quintals grain to ration shop F.
\item A supplies 50 quintals grain to ration shop D.
\item A supplies 0 quintals grain to ration shop E.
\item A supplies 0 quintals grain to ration shop F.
\end{enumerate}
