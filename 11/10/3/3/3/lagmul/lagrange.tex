\iffalse
\documentclass[journal,12pt,twocolumn]{IEEEtran}
\usepackage{romannum}
\usepackage{float}
\usepackage{setspace}
\usepackage{gensymb}
\singlespacing
\usepackage[cmex10]{amsmath}
\usepackage{amsthm}
\usepackage{mathrsfs}
\usepackage{txfonts}
\usepackage{stfloats}
\usepackage{bm}
\usepackage{cite}
\usepackage{cases}
\usepackage{subfig}
\usepackage{longtable}
\usepackage{multirow}
\usepackage{enumitem}
\usepackage{mathtools}
\usepackage{steinmetz}
\usepackage{tikz}
\usepackage{circuitikz}
\usepackage{verbatim}
\usepackage{tfrupee}
\usepackage[breaklinks=true]{hyperref}
\usepackage{tkz-euclide}
\usetikzlibrary{calc,math}
\usepackage{listings}
    \usepackage{color}                                            %%
    \usepackage{array}                                            %%
    \usepackage{longtable}                                        %%
    \usepackage{calc}                                             %%
    \usepackage{multirow}                                         %%
    \usepackage{hhline}                                           %%
    \usepackage{ifthen}                                           %%
  %optionally (for landscape tables embedded in another document): %%
    \usepackage{lscape}     
\usepackage{multicol}
\usepackage{chngcntr}
\DeclareMathOperator*{\Res}{Res}
\renewcommand\thesection{\arabic{section}}
\renewcommand\thesubsection{\thesection.\arabic{subsection}}
\renewcommand\thesubsubsection{\thesubsection.\arabic{subsubsection}}

\renewcommand\thesectiondis{\arabic{section}}
\renewcommand\thesubsectiondis{\thesectiondis.\arabic{subsection}}
\renewcommand\thesubsubsectiondis{\thesubsectiondis.\arabic{subsubsection}}

% correct bad hyphenation here
\hyphenation{op-tical net-works semi-conduc-tor}
\def\inputGnumericTable{}                                 %%

\lstset{
frame=single, 
breaklines=true,
columns=fullflexible
}

\begin{document}


\newtheorem{theorem}{Theorem}[section]
\newtheorem{problem}{Problem}
\newtheorem{proposition}{Proposition}[section]
\newtheorem{lemma}{Lemma}[section]
\newtheorem{corollary}[theorem]{Corollary}
\newtheorem{example}{Example}[section]
\newtheorem{definition}[problem]{Definition}
\newcommand{\BEQA}{\begin{eqnarray}}
\newcommand{\EEQA}{\end{eqnarray}}
\newcommand{\define}{\stackrel{\triangle}{=}}

\bibliographystyle{IEEEtran}
\providecommand{\mbf}{\mathbf}
\providecommand{\pr}[1]{\ensuremath{\Pr\left(#1\right)}}
\providecommand{\qfunc}[1]{\ensuremath{Q\left(#1\right)}}
\providecommand{\sbrak}[1]{\ensuremath{{}\left[#1\right]}}
\providecommand{\lsbrak}[1]{\ensuremath{{}\left[#1\right.}}
\providecommand{\rsbrak}[1]{\ensuremath{{}\left.#1\right]}}
\providecommand{\brak}[1]{\ensuremath{\left(#1\right)}}
\providecommand{\lbrak}[1]{\ensuremath{\left(#1\right.}}
\providecommand{\rbrak}[1]{\ensuremath{\left.#1\right)}}
\providecommand{\cbrak}[1]{\ensuremath{\left\{#1\right\}}}
\providecommand{\lcbrak}[1]{\ensuremath{\left\{#1\right.}}
\providecommand{\rcbrak}[1]{\ensuremath{\left.#1\right\}}}
\theoremstyle{remark}
\newtheorem{rem}{Remark}
\newcommand{\sgn}{\mathop{\mathrm{sgn}}}
\providecommand{\abs}[1]{\left\vert#1\right\vert}
\providecommand{\res}[1]{\Res\displaylimits_{#1}} 
\providecommand{\norm}[1]{\left\lVert#1\right\rVert}
\providecommand{\mtx}[1]{\mathbf{#1}}
\providecommand{\mean}[1]{E\left[ #1 \right]}
\providecommand{\fourier}{\overset{\mathcal{F}}{ \rightleftharpoons}}
\providecommand{\system}{\overset{\mathcal{H}}{ \longleftrightarrow}}
\newcommand{\solution}{\noindent \textbf{Solution: }}
\newcommand{\cosec}{\,\text{cosec}\,}
\providecommand{\dec}[2]{\ensuremath{\overset{#1}{\underset{#2}{\gtrless}}}}
\newcommand{\myvec}[1]{\ensuremath{\begin{pmatrix}#1\end{pmatrix}}}
\newcommand{\mydet}[1]{\ensuremath{\begin{vmatrix}#1\end{vmatrix}}}
\numberwithin{equation}{subsection}
\makeatletter
\@addtoreset{figure}{problem}
\makeatother

\let\StandardTheFigure\thefigure
\let\vec\mathbf
\renewcommand{\thefigure}{\theproblem}



\def\putbox#1#2#3{\makebox[0in][l]{\makebox[#1][l]{}\raisebox{\baselineskip}[0in][0in]{\raisebox{#2}[0in][0in]{#3}}}}
     \def\rightbox#1{\makebox[0in][r]{#1}}
     \def\centbox#1{\makebox[0in]{#1}}
     \def\topbox#1{\raisebox{-\baselineskip}[0in][0in]{#1}}
     \def\midbox#1{\raisebox{-0.5\baselineskip}[0in][0in]{#1}}

\vspace{3cm}


\title{Assignment 5}
\author{Jaswanth Chowdary Madala}


% make the title area
\maketitle

\newpage

%\tableofcontents

\bigskip

\renewcommand{\thefigure}{\theenumi}
\renewcommand{\thetable}{\theenumi}

\begin{enumerate}
\item Find the perpendicular distance from the origin to the line $x – y = 4$ and angle between perpendicular and the positive x-axis.\\
\textbf{Solution:} 
\fi
		The given problem can be expressed as a constrained optimization problem as 
\begin{align}
	\min_{\vec{x}}f\brak{\vec{x}} &\triangleq \norm{\vec{x}-\vec{P}}^2 \label{eq:11/10/3/3/3/lagmul/1} \\
	\text{s.t.} \quad g\brak{\vec{x}} &\triangleq \vec{n}^T\vec{x}-c = 0 \label{eq:11/10/3/3/3/lagmul/2}
\end{align}
where
\begin{align}
\vec{P} = \myvec{0\\0}, \, \vec{n} = \myvec{1\\-1}, \, c &= 4 \label{eq:11/10/3/3/3/lagmul/params}
\end{align}
Define
\begin{align}
L\brak{\vec{x},\lambda} &= f\brak{\vec{x}} - \lambda g\brak{\vec{x}}
\end{align}
Here we find the optimal point, $\vec{Q}$ that is closest to the point $\vec{P}$, by finding $\lambda$ using the following equation
\begin{align}
\nabla L\brak{\vec{x},\lambda} &= 0 \label{eq:11/10/3/3/3/lagmul/3}
\end{align}
Here we have, 
\begin{align}
\nabla f\brak{\vec{x}} &= 2\brak{\vec{x}-\vec{P}} \label{eq:11/10/3/3/3/lagmul/4}\\
\nabla g\brak{\vec{x}} &= \vec{n}\label{eq:11/10/3/3/3/lagmul/5}
\end{align}
From \eqref{eq:11/10/3/3/3/lagmul/3}, \eqref{eq:11/10/3/3/3/lagmul/4}, \eqref{eq:11/10/3/3/3/lagmul/5} we get
\begin{align}
2\brak{\vec{x}-\vec{P}} - \lambda\vec{n} &= 0\\
\implies \vec{x} &= \frac{\lambda}{2}\vec{n}+\vec{P} \label{eq:11/10/3/3/3/lagmul/6}
\end{align}
The point $\vec{x}$ lies on the given line \eqref{eq:11/10/3/3/3/lagmul/2}  \begin{align}
\vec{n}^\top\brak{\frac{\lambda}{2}\vec{n}+\vec{P}} - c &= 0 \\
\implies \lambda &= -\frac{2\brak{\vec{n}^\top\vec{P}-c}}{\norm{\vec{n}}^2} \label{eq:11/10/3/3/3/lagmul/7}
\end{align}
Substituting \eqref{eq:11/10/3/3/3/lagmul/7} in \eqref{eq:11/10/3/3/3/lagmul/6}, we get the optimal point $\vec{Q}$ as
\begin{align}
\vec{Q} &= \vec{P} - \frac{\vec{n}^\top\vec{P}-c}{\norm{\vec{n}}^2}\vec{n} \label{eq:11/10/3/3/3/lagmul/8}
\end{align}
Substituting \eqref{eq:11/10/3/3/3/lagmul/params} in \eqref{eq:11/10/3/3/3/lagmul/7}, \eqref{eq:11/10/3/3/3/lagmul/8} gives
\begin{align}
\lambda &= 4 \\
\vec{Q} &= \myvec{2 \\-2}
\end{align}
Hence the perpendicular distance is given by
\begin{align}
d &= \norm{\vec{Q}-\vec{P}}\\
&= 2\sqrt{2}
\end{align}
